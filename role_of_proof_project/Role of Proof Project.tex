\documentclass[12pt]{article}
\title{Role of Proof Project}
\author{Dr. Hanusch}  %Put your name on your paper.
\date{Due April 7, 2017}

\setlength{\parindent}{0mm}

\usepackage{paralist}
\usepackage{tabto}
\usepackage{graphicx}
\usepackage{amsmath}
\usepackage{amssymb}
\usepackage{amsthm}
\usepackage{enumitem}
\usepackage[margin=1in]{geometry}
%\usepackage[shortlabels]{enumerate}
\usepackage{setspace} 
\doublespacing

\begin{document}

% ----------------------------------------------------------------
\vfuzz2pt % Don't report over-full v-boxes if over-edge is small
\hfuzz2pt % Don't report over-full h-boxes if over-edge is small
% THEOREMS -------------------------------------------------------
\newtheorem{thm}{Theorem}[section]
\newtheorem{cor}[thm]{Corollary}
\newtheorem{lem}[thm]{Lemma}
\newtheorem{prop}[thm]{Proposition}
\theoremstyle{definition}
\newtheorem{defn}[thm]{Definition}
\newtheorem{qu}[]{Question}
\theoremstyle{remark}
\newtheorem{rem}[thm]{Remark}
\newtheorem{prf}[]{Proof}
\numberwithin{equation}{section}

% MATH -----------------------------------------------------------
\newcommand{\norm}[1]{\left\Vert#1\right\Vert}
\newcommand{\abs}[1]{\left\vert#1\right\vert}
\newcommand{\set}[1]{\left\{#1\right\}}
\newcommand{\Real}{\mathbb R}
\newcommand{\eps}{\varepsilon}
\newcommand{\To}{\longrightarrow}
\newcommand{\BX}{\mathbf{B}(X)}
\newcommand{\A}{\mathcal{A}}

% ----------------------------------------------------------------


\maketitle

\section{Read}

Read Michael deVilliers' 1990 paper, \emph{The Role and Function of Proof in Mathematics.} You can find the full text of this paper on Blackboard.




\section{Writing} 
You will write three components: a synopsis, three proofs and an explanation.

\subsection{Synopsis}

You will write a synopsis of deVilliers' paper, explaining the five roles of proof outlined in the paper. You must be careful not to plagiarize deVilliers' words as your own. Include paragraph citations and a reference list. You may choose either MLA or APA formatting. 

The synopsis must be a minimum of one page and a maximum of two pages typed in 12 point, double-spaced with 1 inch margins. This \LaTeX document has been given the necessary commands for that formatting.


\subsection{Proofs}
Choose a theorem, it may be your own theorem, one from the homework, one from the textbook, or from another source, and cite the source of the theorem (included in the reference list, unless you make it up yourself or take it from the homework). Then write three distinct proofs of the statement. Each proof must highlight a different role of proof from the deVilliers paper.


\subsection{Explanation}
Describe the explicit features in each of your proofs that highlight the particular role of the proof.  The explanations of all three proofs combined should be a minimum of one page and a  maximum of two pages typed in 12 point, double-spaced with 1 inch margins. 


\section{Rubric}

\begin{tabular}{|p{4in}|p{1.25in}||p{1.25in}|}
\hline
Feature & Maximum Score & Your Score \\
\hline
Comprehension of deVilliers' paper is demonstrated through the synopsis. & 25 points & \vspace{30pt} \\
\hline
Proof 1 is valid and fluent.  & 10 points &\vspace{30pt} \\
\hline
Proof 1 demonstrates the role of proof. & 5 points &\vspace{30pt} \\
\hline
Proof 2 is valid and fluent. & 10 points &\vspace{30pt} \\
\hline
Proof 2 demonstrates the role of proof. & 5 points &\vspace{30pt} \\
\hline
Proof 3 is valid and fluent. & 10 points &\vspace{30pt} \\
\hline
Proof 3 demonstrates the role of proof. & 5 points &\vspace{30pt} \\
\hline
How each proof served its particular role is explained in detail. & 20 points &\vspace{30pt} \\
\hline
The text is grammatically correct and the language is coherent and flows. & 10 points &\vspace{30pt} \\
\hline

Total  & 100 points & \vspace{30pt}\\ 
\hline
\end{tabular}





\end{document}
