\documentclass[12pt]{article}
\title{Homework 6}
\author{Dr. Hanusch}  %Put your name on your paper.
\date{February 27, 2017}

\setlength{\parindent}{0mm}

\usepackage{paralist}
\usepackage{tabto}
\usepackage{graphicx}
\usepackage{amsmath}
\usepackage{amssymb}
\usepackage{amsthm}
\usepackage{enumitem}
\usepackage{framed}
\usepackage{mathrsfs}
%\usepackage[shortlabels]{enumerate}

\begin{document}

% ----------------------------------------------------------------
\vfuzz2pt % Don't report over-full v-boxes if over-edge is small
\hfuzz2pt % Don't report over-full h-boxes if over-edge is small
% THEOREMS -------------------------------------------------------
\newtheorem{thm}{Theorem}[section]
\newtheorem{cor}[thm]{Corollary}
\newtheorem{lem}[thm]{Lemma}
\newtheorem{prop}[thm]{Proposition}
\theoremstyle{definition}
\newtheorem{defn}[thm]{Definition}
\newtheorem{qu}[]{Question}
\theoremstyle{remark}
\newtheorem{rem}[thm]{Remark}
\newtheorem*{prf}{Proof}
\numberwithin{equation}{section}

% MATH -----------------------------------------------------------
\newcommand{\norm}[1]{\left\Vert#1\right\Vert}
\newcommand{\abs}[1]{\left\vert#1\right\vert}
\newcommand{\set}[1]{\left\{#1\right\}}
\newcommand{\Real}{\mathbb R}
\newcommand{\Zee}{\mathbb Z}
\newcommand{\eps}{\varepsilon}
\newcommand{\To}{\longrightarrow}
\newcommand{\BX}{\mathbf{B}(X)}
\newcommand{\A}{\mathcal{A}}
\newcommand{\U}{\mathcal{U}}
\newcommand{\power}{\mathscr{P}}


% ----------------------------------------------------------------


\maketitle

\section{Read}

Sections 3.4-3.6 and 5.1-5.2 of Kwong. Complete the hands-on exercises.

%% You can delete this section before typing up your turn-in homework%%
\section{Boardwork} 

Due March 1, 2017.

%\begin{qu}
%\NumTabs{3}
%\begin{inparaenum}[a)]
%\item $a \in {a}$
%\tab\item  $\emptyset \in \emptyset$
%\tab\item $2 \in (2,7)$
%\tab\item $\sqrt{5} \in (1,3)$
%\tab\item $|\set{-3, -2, 2, 3}|=4$
%\tab\item $|\set{x\in \mathbb{Q} | x^2=3 }|=2$
%\end{inparaenum}
%\end{qu}

\begin{qu}
Write a useful negation for the following definitions. As an example, in the first question you are saying ``An integer $n$ is NOT a multiple of 3 if and only if ...''.
\begin{enumerate}[label=\alph*)]
\item An integer $n$ is a multiple of 3 if and only if there exists an integer $k$ such that $n=3k$.
\item A set $X$ is a subset of a set $Y$ if and only if for all elements of $X$, $x \in X$ implies $x \in Y$.
\item An integer $p$ is prime if and only if $p|a$ or $p|b$, whenever $p|(ab)$.
\end{enumerate}
\end{qu}

\begin{qu}
Prove that for all real numbers $x$, there exists a real number $y$ such that $x+2y=0$.
\end{qu}

\begin{qu}
Let the universal set be $\Zee$. For each $n \in \Zee$, define $A_n= n\Zee$. Let $P$ be the set of prime numbers. Evaluate $\bigcup_{i \in 5\Zee} A_i$ and $\bigcap_{p \in P} A_p$
\end{qu}

\begin{qu}
Evaluate $\bigcap_{x\in (1,2)} (1-2x, x^2)$ and $\bigcup_{x\in (0,1)} (x, \frac{1}{x})$.
\end{qu}

\begin{qu}
Disprove the following statements by constructing a counterexample.
\begin{enumerate}[label=\alph*)]
\item If $a$ and $b$ are integers with $a|b$, then $a\leq b$.
\item If $a$, $b$ and $c$ are positive integers then $a^{(b^c)}=(a^b)^c$.
\item Let $ T = \set{ (p, q, r) | p=x^2-y^2, q=2xy, \textrm{ and } r=x^2+y^2 \textrm{ where } x, y \in \mathbb{Z} }$ and $P=\set{ (a,b,c) | a,b,c \in \mathbb{Z} \textrm{ and } a^2+b^2=c^2 }$. Prove that $T = P$.
\item If $A=B-C$, then $B=A\cup C$.
\end{enumerate}
\end{qu}



\begin{qu}
Let the universal set be $\mathbb{R}^2$. For each $r \in (0, \infty)$, define
$$A_r=\set{(x,y)| y=rx^2}$$
that is, $A_r$ is the set of points on the parabola $y=rx^2$, where $r$ is a positive real number. Evaluate
$ \bigcap_{r\in (0, \infty)} A_r $ and $ \bigcup_{r\in (0, \infty)} A_r $.
\end{qu}

\begin{qu}
Prove by induction that for all $n \in \mathbb{N}$ that
$$ 1^3 + 2^3 + \dots + n^3 = \frac{n(3n-1)}{2}.$$
\end{qu}

\begin{qu}
Prove by induction that for all $n \in \mathbb{N}$ that
$$ \sum_{i=1}^n \frac{1}{i(i+1)} = 1-\frac{1}{n+1}.$$
\end{qu}

\begin{qu}
Prove by induction that for all $n \in \mathbb{N}$ that
$$ n! \leq n^n.$$
\end{qu}

\begin{qu}
Prove by induction that $n(n+1)(n+2)$ is a multiple of $3$ for all integers $n\geq 1$.
\end{qu}

\begin{qu}
Let $$T_n = \sum_{i=0}^n \frac{1}{(2i+1)(2i+3)}.$$

\begin{enumerate}[label=\alph*)]
\item Evaluate $T_n$ for $n = 1, 2, 3, 4, 5$.
\item Propose a simple formula for $S_n$.
\item Use induction to prove your conjecture for all integers $n \geq 1$.
\end{enumerate}
\end{qu}

\newpage
\section{Turn-in} 

Due March 3, 2017.

%% Stop deleting here.

\begin{qu}
This question is only for students in the 11:30 section.\\

Let $\U$ be a universal set. Let $A_i \subseteq \U$ be a family of sets with $i \in I$ for some index set $I$.
\begin{enumerate}[label=\alph*)]
\item Suppose $B \subseteq A_i$ for every $i \in I$. Prove that $B \subseteq \bigcap_{i \in I} A_i$.
\item Suppose $A_i \subseteq D$ for every $i \in I$. Prove that $\bigcup_{i \in I} A_i \subseteq D$.
\end{enumerate}
\end{qu}

\begin{qu}
Prove the generalized DeMorgan Law, or in other words that for any nonempty index set $I$

$$\overline{\bigcap_{i \in I} A_i} = \bigcup_{i\in I} \overline{A_i}$$
\end{qu}

%Put your answer to the question here, without the comment.
% \newpage 
% Make sure to put your name on the new page

\begin{qu}
Use induction to prove that for all positive integers $n$,
$$ \sum_{i=1}^n i(i+1)(i+2) = \frac{n(n+1)(n+2)(n+3)}{4}.$$
\end{qu}

\begin{qu}
Let $$S_n = \frac{1}{2!} + \frac{2}{3!} + \frac{3}{4!} + \dots + \frac{n}{(n+1)!}.$$

\begin{enumerate}[label=\alph*)]
\item Evaluate $S_n$ for $n = 1, 2, 3, 4, 5$.
\item Propose a simple formula for $S_n$.
\item Use induction to prove your conjecture for all integers $n \geq 1$.
\end{enumerate}
\end{qu}

\end{document}
