\documentclass[12pt]{article}
\title{Homework 11}
\author{Dr. Hanusch}  %Put your name on your paper.
\date{April 17, 2017}

\setlength{\parindent}{0mm}
\usepackage[left=1.5in, right=1.25in, top=1.25in, bottom=1.25in]{geometry}
\usepackage{paralist}
\usepackage{tabto}
\usepackage{graphicx}
\usepackage{amsmath}
\usepackage{amssymb}
\usepackage{amsthm}
\usepackage{enumitem}
\usepackage{framed}
\usepackage{mathrsfs}
%\usepackage{mathtools}
%\usepackage[shortlabels]{enumerate}

\begin{document}

% ----------------------------------------------------------------
\vfuzz4pt % Don't report over-full v-boxes if over-edge is small
\hfuzz4pt % Don't report over-full h-boxes if over-edge is small
% THEOREMS -------------------------------------------------------
\newtheorem{thm}{Theorem}
\newtheorem{cor}[thm]{Corollary}
\newtheorem{lem}[thm]{Lemma}
\newtheorem{prop}[thm]{Proposition}
\theoremstyle{definition}
\newtheorem{defn}[thm]{Definition}
\newtheorem{qu}[]{Question}
\theoremstyle{remark}
\newtheorem{rem}[thm]{Remark}
\newtheorem*{prf}{Proof}
%\numberwithin{equation}{equation}

% MATH -----------------------------------------------------------
\newcommand{\norm}[1]{\left\Vert#1\right\Vert}
\newcommand{\abs}[1]{\left\vert#1\right\vert}
\newcommand{\set}[1]{\left\{#1\right\}}
\newcommand{\Real}{\mathbb R}
\newcommand{\Z}{\mathbb Z}
\newcommand{\N}{\mathbb N}
\newcommand{\eps}{\varepsilon}
\newcommand{\To}{\longrightarrow}
\newcommand{\BX}{\mathbf{B}(X)}
\newcommand{\A}{\mathcal{A}}
\newcommand{\U}{\mathcal{U}}
\newcommand{\power}{\mathscr{P}}
\newcommand{\dv}{\textrm{ div }}


% ----------------------------------------------------------------


\maketitle

\section{Read}

Read sections 7.1-7.3 and 8.1-8.2 of Kwong and Chapter 6 of Lerma. Complete the hands-on exercises in Kwong.

%% You can delete this section before typing up your turn-in homework%%
\section{Boardwork} 

Due April 19, 2017.

%\begin{qu}
%\NumTabs{3}
%\begin{inparaenum}[a)]
%\item $a \in {a}$
%\tab\item  $\emptyset \in \emptyset$
%\tab\item $2 \in (2,7)$
%\tab\item $\sqrt{5} \in (1,3)$
%\tab\item $|\set{-3, -2, 2, 3}|=4$
%\tab\item $|\set{x\in \mathbb{Q} | x^2=3 }|=2$
%\end{inparaenum}
%\end{qu}

\begin{qu}\label{one}
Represent each of the following relations from $\set{1,2,3,6}$ to $\set{1,2,3,6}$ using a digraph and an incidence matrix.
\begin{enumerate}
\item $A_1=\set{(x,y) \mid x \ne y}$
\item $A_2=\set{(x,y) \mid x < y}$
\item $A_3=\set{(x,y) \mid  x|y}$
\item $A_4=\set{(x,y) \mid x+y \textrm{ is even}}$
\end{enumerate}
\end{qu}

\begin{qu}
Find the domain and range of each relation in Question \ref{one}. \end{qu}

\begin{qu}
Prove or disprove that each relation in Question \ref{one} is reflexive.
\end{qu}

\begin{qu}
Prove or disprove that each relation in Question \ref{one} is irreflexive.
\end{qu}

\begin{qu}
Prove or disprove that each relation in Question \ref{one} is symmetric.
\end{qu}

\begin{qu}
Prove or disprove that each relation in Question \ref{one} is antisymmetric.
\end{qu}

\begin{qu}
Prove or disprove that each relation in Question \ref{one} is transitive.
\end{qu}

\begin{qu}
For the relation $S=\set{(x,y) \mid xy \textrm{ is even}}$ on $\N$, prove or disprove that $S$ is reflexive, irreflexive, symmetric, antisymmetric and transitive.
\end{qu}

\begin{qu}
For the relation $T=\set{(x,y) \mid 3 \textrm{ divides } x+2y}$ on $\Z$, prove or disprove that $T$ is reflexive, irreflexive, symmetric, antisymmetric and transitive.
\end{qu}

\begin{qu}
Which, if any, of $A_1, A_2, A_3, A_4, S,$ and $T$ are equivalence relations? \end{qu}

\begin{qu}
Show that the relation $\sim$ on $\Z$, defined by $m \sim n \iff m+n \textrm{ is even}$ is an equivalence relation. Also, describe the equivalence classes.
\end{qu}

\begin{qu}
Let $A_i \set{x \in \Real \mid i \leq x < i+1}$ where $i \in \Z$. Prove that the $A_i$ form a partition of $\Real$.
\end{qu}

\begin{qu} Let $R$ be a relation on $\Real$ such that $a R b$ if and only if $a-b=n$ for some $n \in \Z$. Prove that $R$ is an equivalence relation. \end{qu}

\begin{qu}
Let $R$ be an equivalence relation on $A$, and let $a, x, y \in A$. Prove that if $x, y \in [a]$, then $x R y$.
\end{qu}

\begin{qu} \label{last}
Let $A=\set{x \in \N \mid 2\leq x \leq 100}$. For $p$ a prime number let $A_p = \set{a \in A \mid p \textrm{ is the smallest prime that divides } a }$. Prove these form a partition on $A$.  \end{qu}

\newpage
\section{Turn-in} 

Due Friday April 21, 2017.

%% Stop deleting here.

\begin{qu}
For the relation $V=\set{(x,y) \mid xy>0}$ on $\Z$, prove or disprove that $V$ is reflexive, irreflexive, symmetric, antisymmetric and transitive.

\end{qu}


%Put your answer to the question here, without the comment.
% \newpage 
% Make sure to put your name on the new page

\begin{qu}
Prove Theorem 7.3.1. If $\sim$ is an equivalence relation on $A$, then $a \sim b \iff [a] = [b]$.
\end{qu}

\begin{qu}
Let $A=\set{x \in \N \mid 2\leq x \leq 100}$. For $p$ a prime number let $A_p = \set{a \in A \mid p \textrm{ is the smallest prime that divides } a }$. Note you proved these form a partition on $A$ in Question \ref{last}. 

Define a relation $\sim$ on $A$, such that $a \sim b$ is $a \in A_p$ and $b \in A_p$ for some prime $p$. Prove that $\sim$ is an equivalence relation.
\end{qu}

\end{document}
