\documentclass[12pt]{article}
\title{Homework 5}
\author{Dr. Hanusch}  %Put your name on your paper.
\date{February 20, 2017}

\setlength{\parindent}{0mm}

\usepackage{paralist}
\usepackage{tabto}
\usepackage{graphicx}
\usepackage{amsmath}
\usepackage{amssymb}
\usepackage{amsthm}
\usepackage{enumitem}
\usepackage{framed}
\usepackage{mathrsfs}
%\usepackage[shortlabels]{enumerate}

\begin{document}

% ----------------------------------------------------------------
\vfuzz2pt % Don't report over-full v-boxes if over-edge is small
\hfuzz2pt % Don't report over-full h-boxes if over-edge is small
% THEOREMS -------------------------------------------------------
\newtheorem{thm}{Theorem}[section]
\newtheorem{cor}[thm]{Corollary}
\newtheorem{lem}[thm]{Lemma}
\newtheorem{prop}[thm]{Proposition}
\theoremstyle{definition}
\newtheorem{defn}[thm]{Definition}
\newtheorem{qu}[]{Question}
\theoremstyle{remark}
\newtheorem{rem}[thm]{Remark}
\newtheorem*{prf}{Proof}
\numberwithin{equation}{section}

% MATH -----------------------------------------------------------
\newcommand{\norm}[1]{\left\Vert#1\right\Vert}
\newcommand{\abs}[1]{\left\vert#1\right\vert}
\newcommand{\set}[1]{\left\{#1\right\}}
\newcommand{\Real}{\mathbb R}
\newcommand{\Zee}{\mathbb Z}
\newcommand{\eps}{\varepsilon}
\newcommand{\To}{\longrightarrow}
\newcommand{\BX}{\mathbf{B}(X)}
\newcommand{\A}{\mathcal{A}}
\newcommand{\U}{\mathcal{U}}
\newcommand{\power}{\mathscr{P}}


% ----------------------------------------------------------------


\maketitle

\section{Read}

Sections 4.3-4.5 and 3.4-3.6 of Kwong. Complete the hands-on exercises.

%% You can delete this section before typing up your turn-in homework%%
\section{Boardwork} 

Due February 22, 2017.

%\begin{qu}
%\NumTabs{3}
%\begin{inparaenum}[a)]
%\item $a \in {a}$
%\tab\item  $\emptyset \in \emptyset$
%\tab\item $2 \in (2,7)$
%\tab\item $\sqrt{5} \in (1,3)$
%\tab\item $|\set{-3, -2, 2, 3}|=4$
%\tab\item $|\set{x\in \mathbb{Q} | x^2=3 }|=2$
%\end{inparaenum}
%\end{qu}

\begin{qu}
Let $\U = \Zee$,  $A=2\Zee$, $B=3\Zee$, $C=4\Zee$ and $D=6\Zee$.
Describe the following sets.

\NumTabs{2}
\begin{inparaenum}[a)]
\item $A \cap B$
\tab \item $A \cap C$
\tab \item $A \cup D$
\tab \item $C-A$
\tab \item $A-C$
\tab \item $A\cap \overline{B}$
\tab \item $(A\cap B)\cup C $
\tab \item $(A \cup B)-D$
\end{inparaenum}
\end{qu}

\begin{qu}
Let $X=\set{-2, 2}$, $Y=\set{0, 4}$, and $Z=\set{-3, 0, 3}$.
Find

\NumTabs{2}
\begin{inparaenum}[a)]
\item $X \times Y$ 
\tab \item $X \times X$
\tab \item $X \times Y \times Z$
\tab \item $Y \cap Z$
\tab \item $|Y\cup Z|$
\tab \item $|X \times Y \times Z \times X|$
\end{inparaenum}

Hint: For the last two, you should \underline{not} list the elements of the set before computing the cardinality.

\end{qu}

\begin{qu} 
Prove DeMorgan's laws for sets.
\begin{enumerate}[label=\alph*)]
\item $ \overline{A\cup B} = \overline{A} \cap \overline{B}$
\item $\overline{A \cap B} = \overline{A} \cup \overline{B}$
\end{enumerate}
\end{qu}


\begin{qu} 
Prove the associative properties for sets.
\begin{enumerate}[label=\alph*)]
\item $ (A\cup B)\cup C = A \cup (B \cup C)$
\item $(A \cap B) \cap C = A \cap (B \cap C)$
\end{enumerate}
\end{qu}



\begin{qu}
For any two sets $A$ and $B$, we have $A \subseteq B \iff \overline{B} \subseteq \overline{A}$.
\end{qu}

\begin{qu}
Let $A$, $B$ and $C$ be sets in a universe $\U$. Prove that if $A \subseteq C$ and $B \subseteq C$, then $A\cup B \subseteq C$.
\end{qu}

\begin{qu}
Prove or disprove each of the following statements about arbitrary sets $A$ and $B$. 
\begin{enumerate}[label=\alph*)]
\item $\power(A\cap B)=\power(A) \cap \power(B)$
\item $\power(A\cup B)=\power(A) \cup \power(B)$
\item $\power(A-B) = \power(A) - \power(B)$
\end{enumerate}
\end{qu}

\begin{qu}
Find $\set{a, b, c} \times \power(\set{d})$.
\end{qu}

\begin{qu}
Let $A$, $B$ and $C$ be sets. Prove that if $A \subseteq B$, then $A\times C \subseteq B \times C$.
\end{qu}

\begin{qu}
Let $A_n=[-n, 1+1/n)$ and $B_n=\left(\frac{n-1}{n}, \frac{n+1}{n}\right)$. 
\begin{enumerate}[label=\alph*)]
\item Determine $\bigcup_{n=1}^{\infty} A_n$.
\item Determine $\bigcap_{n=1}^{\infty} A_n$.
\item Prove or disprove that $\bigcap_{n \in \mathbb{N}} B_n =\set{1}$.
\end{enumerate}
\end{qu}



\begin{qu}
Let $\U$ be a universal set. Let $A_i \subseteq \U$ be a family of sets with $i \in I$ for some index set $I$, and let $B \subseteq \U$ be another set. For each statement, either prove the statement or give a counterexample.
\begin{enumerate}[label=\alph*)]
\item $$B \cup \left( \bigcap_{i \in I} A_i \right) =  \bigcap_{i \in I} \left(B \cup A_i \right)$$
\item $$ \left( \bigcup_{i \in I} A_i \right)-B =  \bigcup_{i \in I} \left(A_i -B \right)$$
\end{enumerate}
\end{qu}

\newpage
\section{Turn-in} 

Due February 24, 2017.

%% Stop deleting here.


\begin{qu}
Let $A$ and $B$ be sets in a universe $\U$.
Prove or disprove: $A \cup B = A\cap B$ if and only if $A=B$.

\end{qu}

%Put your answer to the question here, without the comment.
% \newpage 
% Make sure to put your name on the new page

\begin{qu}
Let $A$, $B$ and $C$ be sets in a universe $\U$. Prove or disprove that
\begin{enumerate}[label=\alph*)]
\item $A-B=A\cap \overline{B}$
\item $A-(B-C)=A\cap (\overline{B} \cup C)$
\item $A-(B-C)=(A-B)-C$
\end{enumerate}
\end{qu}

\begin{qu}
Let $\U$ be a universal set. Let $A_i \subseteq \U$ be a family of sets with $i \in I$ for some index set $I$.
\begin{enumerate}[label=\alph*)]
\item Suppose $B \subseteq A_i$ for every $i \in I$. Prove that $B \subseteq \bigcap_{i \in I} A_i$.
\item Suppose $A_i \subseteq D$ for every $i \in I$. Prove that $\bigcup_{i \in I} A_i \subseteq D$.
\end{enumerate}
\end{qu}

\end{document}
