\documentclass[12pt]{article}
\title{Homework 3}
\author{Dr. Hanusch}  %Put your name on your paper.
\date{February 6, 2017}

\setlength{\parindent}{0mm}

\usepackage{paralist}
\usepackage{tabto}
\usepackage{graphicx}
\usepackage{amsmath}
\usepackage{amssymb}
\usepackage{amsthm}
\usepackage{enumitem}
\usepackage{framed}
%\usepackage[shortlabels]{enumerate}

\begin{document}

% ----------------------------------------------------------------
\vfuzz2pt % Don't report over-full v-boxes if over-edge is small
\hfuzz2pt % Don't report over-full h-boxes if over-edge is small
% THEOREMS -------------------------------------------------------
\newtheorem{thm}{Theorem}[section]
\newtheorem{cor}[thm]{Corollary}
\newtheorem{lem}[thm]{Lemma}
\newtheorem{prop}[thm]{Proposition}
\theoremstyle{definition}
\newtheorem{defn}[thm]{Definition}
\newtheorem{qu}[]{Question}
\theoremstyle{remark}
\newtheorem{rem}[thm]{Remark}
\newtheorem*{prf}{Proof}
\numberwithin{equation}{section}

% MATH -----------------------------------------------------------
\newcommand{\norm}[1]{\left\Vert#1\right\Vert}
\newcommand{\abs}[1]{\left\vert#1\right\vert}
\newcommand{\set}[1]{\left\{#1\right\}}
\newcommand{\Real}{\mathbb R}
\newcommand{\eps}{\varepsilon}
\newcommand{\To}{\longrightarrow}
\newcommand{\BX}{\mathbf{B}(X)}
\newcommand{\A}{\mathcal{A}}

% ----------------------------------------------------------------


\maketitle

\section{Read}

Sections 3.1-3.3 and Sections 4.1-4.2 of Kwong. Complete the hands-on exercises.

%% You can delete this section before typing up your turn-in homework%%
\section{Boardwork}

Due February 8, 2017.

\begin{qu} Determine the truth value of each of the following propositions:
\begin{enumerate}[label=\alph*)]
\item For any prime number $x$, the number $x+1$ is composite.
\item There exists a prime number $x$ such that the number $x+4$ is prime.
\item For any real number $x$, if $x^2$ is an integer, then $x$ is also an integer.
\item For any integer $x$, $x^2$ is also an integer.
\item There exists integers $s$ and $t$ such that $1<s<t<187$ and $st=187$.
\item For every integer $n$, there exists an integer $m$ such that $m>n^2$.
\item For all $x \in \mathbb{Z}$, either $x$ is even, or $x$ is odd.
\end{enumerate}
\end{qu}

\begin{qu} Find the negation (in simplest form) of each formula.
\begin{enumerate}[label=\alph*)]
\item $\forall x < 0 \forall y, z \in \mathbb{R} (y < z \implies xy < xz)$
\item $\forall x \in \mathbb{Z} [ p(x) \lor q(x) ]$
\item There exists a rational number $x$ such that for all integers $y$, either $p(x,y)$ or $r(x,y)$ is true.
\item For all real numbers $x$ and $y$, if $x^3 +x -2=y^3+y-2$, then $x=y$.
\end{enumerate}
\end{qu}

\begin{qu}
Show that given any rational number $x$, there exits an integer $y$ such that $x^2y$ is an integer.
\end{qu}

\begin{qu}
Show that, between any two rational numbers $a$ and $b$, where $a<b$, there exists another rational number. \end{qu}

\begin{qu}
Prove that the polynomial $x^{71}-4x^{44}+11x-3$ has a real root.
\end{qu}




\begin{qu}
Prove that the sum of two even numbers is even.
\end{qu}

\begin{qu}
Prove that if $m$ is even and $n$ is odd, then $mn$ is even.
\end{qu}

\begin{qu}
Prove that for integers $a$, $b$ and $c$, if $a|b$ and $b|c$, then $a|c$.
\end{qu}

\begin{qu}
Prove that if an integer $x$ is a multiple of 5, then the ones digit of the number is either 0 or 5.

Hint: After you unpack what it means to be a multiple of 5, you are going to need two cases: a case for even and a case for odd.
\end{qu}


\begin{qu} Prove or disprove with a counterexample:\\
For integers $a, b, c$ and $d$, if $a$ divides $b-c$ and $a$ divides $c-d$, then $a$ divides $b-d$.
\end{qu}

\begin{qu}
If $(G, \ast)$ is a cyclic group, then $(G, \ast)$ is abelian.

Write the first and last statements of the proofs in the following proof frameworks:
\begin{enumerate}[label=\alph*)]
\item For a direct proof.
\item $\bigstar$ For a proof by contrapositive.
\item $\bigstar$ For a proof by contradiction.
\end{enumerate}
\end{qu}

\begin{qu} $\bigstar$
Let $x$ and $y$ be two real numbers. Prove that if $x \ne 0$ and $y \ne 0$, then $xy \ne 0$.
\end{qu}

\begin{qu} $\bigstar$
Prove that $\sqrt{3}$ is irrational.
\end{qu}

\begin{qu} $\bigstar$
Prove that $n$ is odd if and only if $n^2$ is odd. \end{qu}

\newpage
\section{Turn-in}

Due February 10, 2017.

%% Stop deleting here.


\begin{qu} For the definition of continuity below: 1) write the second half of the definition using symbols, and 2) negate the statement.

\begin{framed}
A real valued function $f$ is continuous at $a$ if and only if
for every $\epsilon >0$, there is a $\delta >0$ such that $|f(x)-f(a)|<\epsilon$, whenever $|x-a|< \delta$. \end{framed}

\end{qu}

%Put your answer to the question here, without the comment.
% \newpage
% Make sure to put your name on the new page

\begin{qu}
Show that given any rational number $x$, and any positive integer $k$, there exits an integer $y$ such that $x^k y$ is an integer.
\end{qu}

\begin{qu}Suppose $n$ is an odd integer. Prove that $n=4j+1$ for some integer $j$, or $n=4k+3$ for some integer $k$.
\end{qu}

\end{document}
