\documentclass[12pt]{article}
\title{Homework 1}
\author{Dr. Hanusch}  %Put your name on your paper.
\date{January 23, 2017}

\setlength{\parindent}{0mm}

\usepackage{paralist}
\usepackage{tabto}
\usepackage{graphicx}
\usepackage{amsmath}
\usepackage{amssymb}
\usepackage{amsthm}
\usepackage{enumitem}
%\usepackage[shortlabels]{enumerate}

\begin{document}

% ----------------------------------------------------------------
\vfuzz2pt % Don't report over-full v-boxes if over-edge is small
\hfuzz2pt % Don't report over-full h-boxes if over-edge is small
% THEOREMS -------------------------------------------------------
\newtheorem{thm}{Theorem}[section]
\newtheorem{cor}[thm]{Corollary}
\newtheorem{lem}[thm]{Lemma}
\newtheorem{prop}[thm]{Proposition}
\theoremstyle{definition}
\newtheorem{defn}[thm]{Definition}
\newtheorem{qu}[]{Question}
\theoremstyle{remark}
\newtheorem{rem}[thm]{Remark}
\newtheorem{prf}[]{Proof}
\numberwithin{equation}{section}

% MATH -----------------------------------------------------------
\newcommand{\norm}[1]{\left\Vert#1\right\Vert}
\newcommand{\abs}[1]{\left\vert#1\right\vert}
\newcommand{\set}[1]{\left\{#1\right\}}
\newcommand{\Real}{\mathbb R}
\newcommand{\eps}{\varepsilon}
\newcommand{\To}{\longrightarrow}
\newcommand{\BX}{\mathbf{B}(X)}
\newcommand{\A}{\mathcal{A}}

% ----------------------------------------------------------------


\maketitle

\section{Read}

Read the syllabus. Read chapter 1 and chapter 2 of Kwong. Complete the hands-on exercises.

%% You can delete this section before typing up your turn-in homework%%
\section{Boardwork} 

Due January 25, 2017.

\begin{qu} What is the location of my office? When are my office hours? \end{qu}
\begin{qu}How should you turn-in your homework? \end{qu}
\begin{qu} Prove that, for any integer $k$, $$ \frac{k^2(k+1)^2}{4} + (k+1)^3 = \frac{(k+1)^2(k+2)^2}{4}$$ \end{qu}
\begin{qu} Prove that, for any distinct real numbers $x$ and $y$, $$ \frac{x^3-y^3}{x-y}=x^2+xy+y^2$$ \end{qu}
\begin{qu} Explain why these sentences are not propositions:
\begin{enumerate}[label=\alph*)]
\item She is the captain of the basketball team.
\item $x+y=12$.
\item $x^2=36$. 
\end{enumerate} \end{qu}

\begin{qu} Write the negation of each of these statements.
\begin{enumerate}[label=\alph*)]
\item $\pi$ is an element of $\mathbb{Z}$.
\item $u$ is a vowel.
\item This statement is both true and false.
\item For any real number $x$, $x^2>0$.
\item There exists a SUNY Oswego student whose name is Beth.
\end{enumerate}
\end{qu}

\begin{qu} Determine the truth values of these statements:

\NumTabs{3}
\begin{inparaenum}[a)]
\item $\sqrt{2} \in \mathbb{Z}$
\tab\item  $-1 \notin \mathbb{Z}^+$
\tab\item $0 \in \mathbb{N}$
\tab\item  $\frac{4/2} \in \mathbb{Q}$
\tab\item $1.75 \in \mathbb{Q}$
\tab\item  $18 \in 3\mathbb{Z}$
\tab \item $2 \in 3\mathbb{Z}$
\tab \item $\sqrt{-4} \in \mathbb{R}$
\end{inparaenum}
\end{qu}

\section{Turn-in} 

Due January 27, 2017.

%% Stop deleting here.

\begin{qu} What do you think is the purpose of you presenting your work, even if it's not correct? \end{qu}

%Put your answer to the question here, without the comment.
% \newpage 
% Make sure to put your name on the new page

\begin{qu} Prove that, for any integer $k$, 
$$ \frac{k(k+1)(k+2)(k+3)}{4} + (k+1)(k+2)(k+3) =\frac{(k+1)(k+2)(k+3)(k+4)}{4}$$
 \end{qu}


\begin{qu} For each of the following statements: 1) determine if the statement is a proposition, 2) determine the truth value of the statement, and 3) write the negation of the statement.
\begin{enumerate}[label=\alph*)]
    \item The integer $2^8-1$ is prime.
    \item There exists a real number $x$ such that $x>5$.
    \item For any real number $x$, $\frac{x}{x}=1$.
\end{enumerate}

\end{qu}

\end{document}
