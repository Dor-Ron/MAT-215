\documentclass[12pt]{article}
\title{Homework 10}
\author{Dr. Hanusch}  %Put your name on your paper.
\date{April 10, 2017}

\setlength{\parindent}{0mm}
\usepackage[left=1.5in, right=1.25in, top=1.25in, bottom=1.25in]{geometry}
\usepackage{paralist}
\usepackage{tabto}
\usepackage{graphicx}
\usepackage{amsmath}
\usepackage{amssymb}
\usepackage{amsthm}
\usepackage{enumitem}
\usepackage{framed}
\usepackage{mathrsfs}
%\usepackage{mathtools}
%\usepackage[shortlabels]{enumerate}

\begin{document}

% ----------------------------------------------------------------
\vfuzz4pt % Don't report over-full v-boxes if over-edge is small
\hfuzz4pt % Don't report over-full h-boxes if over-edge is small
% THEOREMS -------------------------------------------------------
\newtheorem{thm}{Theorem}
\newtheorem{cor}[thm]{Corollary}
\newtheorem{lem}[thm]{Lemma}
\newtheorem{prop}[thm]{Proposition}
\theoremstyle{definition}
\newtheorem{defn}[thm]{Definition}
\newtheorem{qu}[]{Question}
\theoremstyle{remark}
\newtheorem{rem}[thm]{Remark}
\newtheorem*{prf}{Proof}
%\numberwithin{equation}{equation}

% MATH -----------------------------------------------------------
\newcommand{\norm}[1]{\left\Vert#1\right\Vert}
\newcommand{\abs}[1]{\left\vert#1\right\vert}
\newcommand{\set}[1]{\left\{#1\right\}}
\newcommand{\Real}{\mathbb R}
\newcommand{\Z}{\mathbb Z}
\newcommand{\N}{\mathbb N}
\newcommand{\eps}{\varepsilon}
\newcommand{\To}{\longrightarrow}
\newcommand{\BX}{\mathbf{B}(X)}
\newcommand{\A}{\mathcal{A}}
\newcommand{\U}{\mathcal{U}}
\newcommand{\power}{\mathscr{P}}
\newcommand{\dv}{\textrm{ div }}


% ----------------------------------------------------------------


\maketitle

\section{Read}

Sections 6.1-6.7 and 7.1-7.3 of Kwong. Complete the hands-on exercises.

%% You can delete this section before typing up your turn-in homework%%
\section{Boardwork} 

Due April 12, 2017.

%\begin{qu}
%\NumTabs{3}
%\begin{inparaenum}[a)]
%\item $a \in {a}$
%\tab\item  $\emptyset \in \emptyset$
%\tab\item $2 \in (2,7)$
%\tab\item $\sqrt{5} \in (1,3)$
%\tab\item $|\set{-3, -2, 2, 3}|=4$
%\tab\item $|\set{x\in \mathbb{Q} | x^2=3 }|=2$
%\end{inparaenum}
%\end{qu}

\begin{qu}
Prove or disprove that the following functions are one-to-one.
\begin{enumerate}[label=\alph*)]
\item $f:\Real \to \Real$, $f(x)=x^3-2x^2+1$
\item $g: [2,\infty) \to \Real$, $g(x)=x^3-2x^2+1$
\item $f:\Real \to \Real$, $f(x)=\abs{1-3x}$
\item $g: \Z \to \Z$, $g(n)=n^3+1$
\item $f: \N \to \N$, \[f(n)=\begin{cases}
(n+1)/2 & n \textrm{ is odd} \\
n/2 & n \textrm{ is even}
\end{cases} \]
\item $g:\Z_{10} \to \Z_{10}$, $g(n)\equiv 5n \pmod{10}$
\item $f:\Z_{10} \to \Z_{10}$, $f(n)\equiv 3n+5 \pmod{10}$
\item $g:\Z_{8} \to \Z_{12}$, $g(n)\equiv 3n \pmod{12}$
\end{enumerate}
\end{qu}

\begin{qu}
Prove or disprove that the functions from question 1 are onto.
\vspace*{24pt}
\end{qu}

\begin{qu}
Construct a one-to-one function $g: [2, 5) \to (1, 4]$, that is not onto.
\end{qu}

\begin{qu}
Construct a function $h: [2, 5) \to (1, 4]$ that is onto, but not one-to-one.
\end{qu}


\begin{defn}
A real-valued function $f$ is \emph{increasing} if and only if for any real values $a$ and $b$, $a<b$ implies that $f(a)<f(b)$.
\end{defn}

\begin{defn}
A real-valued function $f$ is \emph{decreasing} if and only if for any real values $a$ and $b$, $a<b$ implies that $f(a)>f(b)$.
\end{defn}

\begin{qu}
Prove that an increasing function is one-to-one.
\end{qu}

\begin{qu}
Prove that a decreasing function is one-to-one.
\end{qu}

\begin{qu}
Construct an increasing function $f: \Real \to \Real$ that is not surjective.
\end{qu}

\begin{qu}
Given $f: A \to B$, prove that  $C_1 \subseteq C_2 \subseteq A$ implies $f(C_1) \subseteq f(C_2)$.
\end{qu}

\begin{qu}
Given $f: A \to B$ and  $C_1, C_2 \subseteq A$, find a counterexample to $f(C_1\cap C_2) \supseteq f(C_1) \cap f(C_2)$.
\end{qu}

\begin{qu}
Given $f: A \to B$, prove that  $D_1 \subseteq D_2 \subseteq B$ implies $f^{-1}(D_1) \subset f^{-1}(D_2)$.
\end{qu}

\begin{qu} Find the inverse function of $f: \Real \to \Real$ defined by 
\[f(x)=\begin{cases}
3x+5 & \textrm{if } x \leq 6\\
5x-7 & \textrm{if } x > 6.
\end{cases} \]
\end{qu}

\begin{qu}
Suppose $f:A \to B$ and $g: B \to C$ are functions. Prove that if both $f$ and $g$ are one-to-one, then $g \circ f$ is also one-to-one.
\end{qu}

\begin{qu}
Suppose $f:A \to B$ and $g: B \to C$ are functions. Prove that if both $f$ and $g$ are onto, then $g \circ f$ is also onto.
\end{qu}

\begin{defn} Given $f:A \to B$, we can define the induced function $f: \power(A) \to \power(B)$ as the image of $X$ for any $X \in \power{A}$.
\end{defn}

\begin{qu}
Let $f: A \to B$.
\begin{enumerate}[label=\alph*)]
\item What condition on $f$ will ensure that the induced function $f: \power(A) \to \power(B)$ is one-to-one?
\item What condition on $f$ will ensure that the induced function $f: \power(A) \to \power(B)$ is onto?
\end{enumerate}
\end{qu}

\newpage
If you chose these problems, come prepared to put your response under the document camera rather than on the board.
\begin{qu}
Consider each of the following student ``proofs''. Assign each submission a grade of A (correct), C (partially correct) or F (invalid) to each. Justify your choice of grade.
\begin{enumerate}[label=\alph*)]
\item Claim: The function $f:\Real \times \Real \to \Real$ given by $f(x,y)=2x-3y$ is a surjection. \\
\emph{``Proof.''} Suppose $(x,y) \in \Real \times \Real$. Then $x \in \Real$, so $2x \in \Real$. Also, $y \in \Real$, so $3y \in \Real$. Therefore, $2x-3y \in \Real$. Thus $f(x,y) \in \Real$, so $f$ is a surjection.
\item Claim: The function $f: [1,\infty) \to (0, \infty)$ defined by $f(x)=\frac{1}{x}$ maps onto $(0, \infty)$. \\
\emph{``Proof.''} Suppose $w \in (0, \infty)$. Choose $x =\frac{1}{w}$. Then $f(x)=\frac{1}{\frac{1}{w}}=w$. Therefore the function $f$ is onto $(0, \infty)$.
\item Claim: The function $f: \Real \to \Real$ given by $f(x)=2x+7$ is onto $\Real$. \\
\emph{``Proof.''} Suppose $f$ is not onto $\Real$. Then there exists $b \in \Real$ with $b \notin \textrm{im}( f)$. Thus for all real numbers $x$, $b \ne 2x+7$. But $a= \frac{1}{2}(b-7)$ is a real number, and $f(a)=b$. This is a contradiction. Thus $f$ is onto $\Real$.
\end{enumerate}
\end{qu}


\begin{qu}
Consider each of the following student ``proofs''. Assign each submission a grade of A (correct), C (partially correct) or F (invalid) to each. Justify your choice of grade.
\begin{enumerate}[label=\alph*)]
\item Claim: The function $f: \Real \to \Real$ given by $f(x)=2x+7$ is one-to-one. \\
\emph{``Proof.''} Suppose $x_1$ and $x_2$ are real numbers with $f(x_1)\ne f(x_2)$. Then $2x_1+7\ne 2x_2+7$ and thus $2x_1 \ne 2x_2$. Hence $x_1 \ne x_2$, which shows that $f$ is one-to-one.
\item Claim: Let $I$ be the interval $(0,1)$. The function $f:I \times I \to I$ given by $f(x,y)=x^y$ is injective. \\
\emph{``Proof.''} Suppose $(x,y)$ and $(x,z)$ are in $I \times I$ and $f(x,y)=f(x,z)$. Then $x^y=x^z$. Dividing by $x^z$, we have $x^{y-x}=x^0=1$. Since $x \ne 1$ and $x^{y-z}=1$, $y-z$ must be 0. Therefore, $y=z$. This shows that $(x,y)=(x,z)$, so $f$ is injective.
\item Claim: Let $I$ be the interval $(0,1)$. The function $f:I \times I \to I$ given by $f(x,y)=x^y$ is not injective. \\
\emph{``Proof.''}  Both $\left(\frac{1}{2}, \frac{1}{2}\right)$ and $\left(\frac{1}{4}, \frac{1}{4}\right)$ are in $I \times I$. But
$$f\left(\frac{1}{4}, \frac{1}{4}\right) = \left(\frac{1}{4}\right)^\frac{1}{4} =\left( \left(\frac{1}{2} \right)^2 \right) ^\frac{1}{4}=\left(\frac{1}{2} \right)^\frac{1}{2}=f\left(\frac{1}{2}, \frac{1}{2}\right).$$
Therefore, $f$ is not injective.
\end{enumerate}
\end{qu}
\newpage
\section{Turn-in} 

Due Monday April 17, 2017.

%% Stop deleting here.

\begin{qu}
Let $f: \Z \times \Z^\ast \to \mathbb Q$ be defined as $f(a,b)=a/b$. 
\begin{enumerate}[label=\alph*)]
\item Prove that $f$ is a well-defined function.
\item Prove or disprove that $f$ is injective.
\item Prove or disprove that $f$ is surjective.
\end{enumerate}

\emph{Note:} $\Z^\ast$ is the set of nonzero integers.
\end{qu}


%Put your answer to the question here, without the comment.
% \newpage 
% Make sure to put your name on the new page

\begin{qu}
Let $f:A \to B$ and let $X \subseteq A$ and $U \subseteq B$. Prove that $f(X) \subseteq U$ if and only if $X\subseteq f^{-1}(U)$
\end{qu}

\begin{qu}
Suppose $f:A \to B$ and $g: B \to C$ are functions. Prove that if both $g$ and $g\circ f$ are one-to-one, then $f$ is also one-to-one.
\end{qu}

\end{document}
