\documentclass[12pt]{article}
\title{Homework 8}
\author{Dr. Hanusch}  %Put your name on your paper.
\date{March 20, 2017}

\setlength{\parindent}{0mm}

\usepackage{paralist}
\usepackage{tabto}
\usepackage{graphicx}
\usepackage{amsmath}
\usepackage{amssymb}
\usepackage{amsthm}
\usepackage{enumitem}
\usepackage{framed}
\usepackage{mathrsfs}
%\usepackage[shortlabels]{enumerate}

\begin{document}

% ----------------------------------------------------------------
\vfuzz2pt % Don't report over-full v-boxes if over-edge is small
\hfuzz2pt % Don't report over-full h-boxes if over-edge is small
% THEOREMS -------------------------------------------------------
\newtheorem{thm}{Theorem}[section]
\newtheorem{cor}[thm]{Corollary}
\newtheorem{lem}[thm]{Lemma}
\newtheorem{prop}[thm]{Proposition}
\theoremstyle{definition}
\newtheorem{defn}[thm]{Definition}
\newtheorem{qu}[]{Question}
\theoremstyle{remark}
\newtheorem{rem}[thm]{Remark}
\newtheorem*{prf}{Proof}
\numberwithin{equation}{section}

% MATH -----------------------------------------------------------
\newcommand{\norm}[1]{\left\Vert#1\right\Vert}
\newcommand{\abs}[1]{\left\vert#1\right\vert}
\newcommand{\set}[1]{\left\{#1\right\}}
\newcommand{\Real}{\mathbb R}
\newcommand{\Zee}{\mathbb Z}
\newcommand{\eps}{\varepsilon}
\newcommand{\To}{\longrightarrow}
\newcommand{\BX}{\mathbf{B}(X)}
\newcommand{\A}{\mathcal{A}}
\newcommand{\U}{\mathcal{U}}
\newcommand{\power}{\mathscr{P}}
\newcommand{\dv}{\textrm{ div }}


% ----------------------------------------------------------------


\maketitle

\section{Read}

Sections 5.1-5.7 of Kwong. Complete the hands-on exercises.

%% You can delete this section before typing up your turn-in homework%%
\section{Boardwork} 

Due March 22, 2017.

%\begin{qu}
%\NumTabs{3}
%\begin{inparaenum}[a)]
%\item $a \in {a}$
%\tab\item  $\emptyset \in \emptyset$
%\tab\item $2 \in (2,7)$
%\tab\item $\sqrt{5} \in (1,3)$
%\tab\item $|\set{-3, -2, 2, 3}|=4$
%\tab\item $|\set{x\in \mathbb{Q} | x^2=3 }|=2$
%\end{inparaenum}
%\end{qu}

\begin{qu}
Prove the transitive property of divisibility, which state that for all integers $a$, $b$, and $c$, where $a\ne 0$, we have that $a|b$ and $b|c$ implies that $a |c$.
\end{qu}

\begin{qu}
Find $b \dv a$ and $b \mod a$, where \\
\NumTabs{3}
\begin{inparaenum}[a)]
\item $a=19, b=79$
\tab\item  $a=59, b=18$
\tab\item $a=16, b=-832$
\tab\item $a=-16, b=172$
\tab\item $a=-8, b=-67$
\tab\item $a=-12, b=-134$
\end{inparaenum}
\end{qu}

\begin{qu}
Use the Euclidean algorithm to determine the greatest common divisor of 1824 and 750.
\end{qu}

\begin{qu}
Prove that 
$$b \mod a \in \set{0, 1, 2, \dots , |a|-1 }$$
for any integers $a$ and $b$, where $a \ne 0$.
\end{qu}

\begin{qu}
Prove that the square of any integer is of the form $3k$ or $3k+1$.
\end{qu}

\begin{qu}
Prove that among any four consecutive integers, one of them is a multiple of four.
\end{qu}

\begin{qu}
Prove that the set $\set{n, n+4, n+8, n+12, n+16}$ contains a multiple of 5 for any positive integer $n$.
\end{qu}

\begin{qu}
Use induction to prove that $3 | (2^{2n}-1)$ for all integers $n \geq 1$.
\end{qu}

\begin{qu}
Use induction to prove that $5 | (3^{3n+1}+2^{n+1})$ for all integers $n \geq 1$.
\end{qu}

\begin{qu}
Prove that any consecutive odd positive integers are relatively prime.
\end{qu}

\newpage
\section{Turn-in} 

Due March 24, 2017.

%% Stop deleting here.

\begin{qu}
Let $a, b,$ and $c$ be integers such that $a \ne 0$. Prove that if $a|b$ and $a|c$, then $a|(sb+tc)$ for any integers $s$ and $t$.
\end{qu}


%Put your answer to the question here, without the comment.
% \newpage 
% Make sure to put your name on the new page

\begin{qu}
Let $m$ and $n$ be positive integers. Prove that $\gcd(m, m+n) | n$.
\end{qu}

\begin{qu}
Look over all of your returned turn-in problems on homeworks 4-7. Select one question that needed improvement and revise and resubmit that proof. Attach your original proof, so I can compare and see your revisions. I am going to grade this revision on a 5 point scale, for correctness, not completion.
\end{qu}

\end{document}
