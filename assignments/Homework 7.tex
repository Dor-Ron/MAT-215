\documentclass[12pt]{article}
\title{Homework 7}
\author{Dr. Hanusch}  %Put your name on your paper.
\date{March 6, 2017}

\setlength{\parindent}{0mm}

\usepackage{paralist}
\usepackage{tabto}
\usepackage{graphicx}
\usepackage{amsmath}
\usepackage{amssymb}
\usepackage{amsthm}
\usepackage{enumitem}
\usepackage{framed}
\usepackage{mathrsfs}
%\usepackage[shortlabels]{enumerate}

\begin{document}

% ----------------------------------------------------------------
\vfuzz2pt % Don't report over-full v-boxes if over-edge is small
\hfuzz2pt % Don't report over-full h-boxes if over-edge is small
% THEOREMS -------------------------------------------------------
\newtheorem{thm}{Theorem}[section]
\newtheorem{cor}[thm]{Corollary}
\newtheorem{lem}[thm]{Lemma}
\newtheorem{prop}[thm]{Proposition}
\theoremstyle{definition}
\newtheorem{defn}[thm]{Definition}
\newtheorem{qu}[]{Question}
\theoremstyle{remark}
\newtheorem{rem}[thm]{Remark}
\newtheorem*{prf}{Proof}
\numberwithin{equation}{section}

% MATH -----------------------------------------------------------
\newcommand{\norm}[1]{\left\Vert#1\right\Vert}
\newcommand{\abs}[1]{\left\vert#1\right\vert}
\newcommand{\set}[1]{\left\{#1\right\}}
\newcommand{\Real}{\mathbb R}
\newcommand{\Zee}{\mathbb Z}
\newcommand{\eps}{\varepsilon}
\newcommand{\To}{\longrightarrow}
\newcommand{\BX}{\mathbf{B}(X)}
\newcommand{\A}{\mathcal{A}}
\newcommand{\U}{\mathcal{U}}
\newcommand{\power}{\mathscr{P}}


% ----------------------------------------------------------------


\maketitle

\section{Read}

Sections 3.6 and 5.1-5.4 of Kwong. Complete the hands-on exercises.

%% You can delete this section before typing up your turn-in homework%%
\section{Boardwork} 

Due March 8, 2017.

%\begin{qu}
%\NumTabs{3}
%\begin{inparaenum}[a)]
%\item $a \in {a}$
%\tab\item  $\emptyset \in \emptyset$
%\tab\item $2 \in (2,7)$
%\tab\item $\sqrt{5} \in (1,3)$
%\tab\item $|\set{-3, -2, 2, 3}|=4$
%\tab\item $|\set{x\in \mathbb{Q} | x^2=3 }|=2$
%\end{inparaenum}
%\end{qu}

Let $F_n$ be the sequence of Fibonacci numbers. These satisfy the following definition: 
$F_0=0, F_1=1$ and $F_n=F_{n-1}+F_{n-2}$ for all integers $n\geq 2$.

\begin{qu}
 Use mathematical induction to prove the identity
 $$F_1^2+F_2^2 + \dots + F_n^2 = F_n F_{n+1}$$
 for any integer $n \geq 1$.
\end{qu}

\begin{qu}
Use induction to prove that for all integers $n\geq 1$:
$$F_1+F_3+\dots +F_{2n-1} = F_{2n}.$$
\end{qu}

\begin{qu}
The sequence $\set{b_n}_{n-1}^\infty$ is defined recursively by 
$$b_n=3b_{n-1}-2$$
for $n\geq 2$, with $b_1=4$. Use induction to prove that $b_n=3^n+1$ for all $n\geq 1$.
\end{qu}

\begin{qu}
The sequence $\set{c_n}_{n-1}^\infty$ is defined recursively by $c_1=3$, $c_2=-9$, and
$$c_n=7c_{n-1}-10c_{n-2}$$
for $n\geq 3$. Use induction to prove that $c_n=4\cdot 2^n-5^n$ for all $n\geq 1$.
\end{qu}

\begin{qu} Write a useful negation to complete the following sentence:

A set $T$ is not well-ordered  \ldots
\vspace*{12pt}
\end{qu}

\begin{qu} Determine which of the following subsets of $\mathbb{R}$ are well-ordered:
\begin{enumerate}[label=\alph*)]
\item $\emptyset$
\item $\set{-9, -7, -6, 5, 11}$
\item $2\mathbb{N}$
\item $2\mathbb{Z}$
\end{enumerate}
\end{qu}

\begin{qu} Prove that the interval $[3,5]$ is not well-ordered. \end{qu}

\begin{qu}
Assume $\emptyset \ne T_1 \subseteq T_2 \subseteq \mathbb{R}$. Prove that if $T_1$ does not have a smallest element, then $T_2$ is not well-ordered.
\end{qu}

\newpage
\section{Turn-in} 

Due March 10, 2017.

%% Stop deleting here.

\begin{qu}
The sequence $\set{b_n}_{n-1}^\infty$ is defined recursively by 
$$b_n=5b_{n-1}+2$$
for $n\geq 2$, with $b_1=1$. Use induction to prove that $b_n=\left(\frac{3}{10}\right)5^n-\frac{1}{2}$ for all $n\geq 1$.
\end{qu}


%Put your answer to the question here, without the comment.
% \newpage 
% Make sure to put your name on the new page

\begin{qu}
Assume $\emptyset \ne T_1 \subseteq T_2 \subseteq \mathbb{R}$. Show that if $T_2$ is well ordered, then $T_1$ is also well-ordered.
\end{qu}

\end{document}
