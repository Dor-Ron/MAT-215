\documentclass[12pt]{article}
\title{Homework 4}
\author{Dr. Hanusch}  %Put your name on your paper.
\date{February 13, 2017}

\setlength{\parindent}{0mm}

\usepackage{paralist}
\usepackage{tabto}
\usepackage{graphicx}
\usepackage{amsmath}
\usepackage{amssymb}
\usepackage{amsthm}
\usepackage{enumitem}
\usepackage{framed}
%\usepackage[shortlabels]{enumerate}

\begin{document}

% ----------------------------------------------------------------
\vfuzz2pt % Don't report over-full v-boxes if over-edge is small
\hfuzz2pt % Don't report over-full h-boxes if over-edge is small
% THEOREMS -------------------------------------------------------
\newtheorem{thm}{Theorem}[section]
\newtheorem{cor}[thm]{Corollary}
\newtheorem{lem}[thm]{Lemma}
\newtheorem{prop}[thm]{Proposition}
\theoremstyle{definition}
\newtheorem{defn}[thm]{Definition}
\newtheorem{qu}[]{Question}
\theoremstyle{remark}
\newtheorem{rem}[thm]{Remark}
\newtheorem*{prf}{Proof}
\numberwithin{equation}{section}

% MATH -----------------------------------------------------------
\newcommand{\norm}[1]{\left\Vert#1\right\Vert}
\newcommand{\abs}[1]{\left\vert#1\right\vert}
\newcommand{\set}[1]{\left\{#1\right\}}
\newcommand{\Real}{\mathbb R}
\newcommand{\eps}{\varepsilon}
\newcommand{\To}{\longrightarrow}
\newcommand{\BX}{\mathbf{B}(X)}
\newcommand{\A}{\mathcal{A}}


% ----------------------------------------------------------------


\maketitle

\section{Read}

Sections 4.1-4.4 of Kwong. Complete the hands-on exercises.

%% You can delete this section before typing up your turn-in homework%%
\section{Boardwork} 

Due February 15, 2017.

\begin{qu}
If $(G, \ast)$ is a cyclic group, then $(G, \ast)$ is abelian. 

Write the first and last statements of the proofs in the following proof frameworks:
\begin{enumerate}[label=\alph*)]
\item For a direct proof.
\item For a proof by contrapositive.
\item For a proof by contradiction.
\end{enumerate}
\end{qu}

\begin{qu} 
Let $p$ be a prime integer. Prove that if $p|a^2$, then $p|a$.
\end{qu}

\begin{qu} 
Let $x$ and $y$ be two real numbers. Prove that if $x \ne 0$ and $y \ne 0$, then $xy \ne 0$.
\end{qu}

\begin{qu} 
Prove that $\sqrt{7}$ is irrational.
\end{qu}

\begin{qu} 
Prove that $n^2$ is a multiple of 3 if and only if $n$ is a multiple of 3. \end{qu}


\begin{qu} Prove or disprove with a counterexample:\\
For integers $a, b$ and $c$, if $a|(bc)$, then $a|b$ or $a|c$. 
\end{qu}

\begin{qu}
Answer each of the following:
\begin{enumerate}[label=\alph*)]
\item Use the roster method to describe $$\set{x\in \mathbb{N}\: | \: x<20 \textrm{ and } x \textrm{ is a multiple of }3 \textrm{ or a multiple of} 5}$$ 
\item Use the roster method to describe $$\set{x\in \mathbb{Z}\: | \: |x|<20 \textrm{ and } x \textrm{ is a multiple of }3\textrm{, but not a multiple of } 5}$$ 
\item Use set-builder notation to describe $$\set{ 0, 4, 8, 12, \dots }$$
\item Describe the set in the previous problem as $S^+$, $S^-$, $bS$ or $a+bS$ for some appropriate set $S$.
\end{enumerate}
\end{qu}

\begin{qu}
Determine which of the following statements are true, and which are false. Justify your choice.

\NumTabs{3}
\begin{inparaenum}[a)]
\item $a \in {a}$
\tab\item  $\emptyset \in \emptyset$
\tab\item $2 \in (2,7)$
\tab\item $\sqrt{5} \in (1,3)$
\tab\item $|\set{-3, -2, 2, 3}|=4$
\tab\item $|\set{x\in \mathbb{Q} | x^2=3 }|=2$
\end{inparaenum}
\end{qu}

\begin{qu}
Determine whether the following statements are correct or incorrect \emph{syntactically}. For those that are syntactically correct, determine their truth values; for those that are syntactically incorrect, suggest ways to fix them.
\begin{enumerate}[label=\alph*)]
\item $(3,7]=3<x\leq 7$.
\item $\set{ x \in \mathbb{R} | x^2<0}\equiv \emptyset $.
\item $\frac{7}{4} \in [2, \sqrt{7})$.
\item If $(0, \infty)$, then $x$ is positive.
\item $5 \subseteq (0, 6]$.
\end{enumerate}
\end{qu}

\begin{qu}
For each of the following statements about sets $A$, $B$, and $C$, either prove the statement is true or give a counterexample to show that it is false.
\begin{enumerate}[label=\alph*)]
\item If $A\subseteq B$ and $B\subseteq C$, then $A \subseteq C$.
\item If $A \in B$ and $B \subseteq C$, then $A \subseteq C$.
\item If $A \in B$ and $B \in C$, then $A \in C$.
\end{enumerate}
\end{qu}

\newpage
\section{Turn-in} 

Due February 17, 2017.

%% Stop deleting here.


\begin{qu}
Prove or disprove with a counterexample the following statement: \\
If the sum of two positive prime numbers is prime, then one of the prime addends must be 2.
\end{qu}

%Put your answer to the question here, without the comment.
% \newpage 
% Make sure to put your name on the new page

\begin{qu}
Let $P$ be the set of Pythagorean triples; that is,
$$ P=\set{ (a,b,c) | a,b,c \in \mathbb{Z} \textrm{ and } a^2+b^2=c^2 } $$

and let $T$ be the set
$$ T = \set{ (p, q, r) | p=x^2-y^2, q=2xy, \textrm{ and } r=x^2+y^2 \textrm{ where } x, y \in \mathbb{Z} }.$$

Prove that $T \subseteq P$.
\end{qu}

\begin{qu}
Read the comments from homework 3 turn-in. Rewrite either Question 16 or Question 17, incorporating the comments. Choose which ever question needed the most revision. Staple the original (with my comments) behind your rewrite.
\end{qu}

\end{document}
